\documentclass{beamer}

\usetheme{Rochester}

\usepackage[latin1]{inputenc}
\usepackage[T1]{fontenc}
\usepackage[french]{babel}
\usepackage{lmodern}

\usepackage{moreverb}

% Informations sur le document
\title{Construire une pr�sentation avec \LaTeX}
\subtitle{Le package Beamer}
\author{S�bastien Comb�fis\inst{1}}
\institute{\inst{1}UKO}
\date{27 novembre 2010}

\begin{document}
	\frame[plain]{
		\titlepage
	}
	
	\section{D�finir un transparent}
	\begin{frame}
		\frametitle{Plan}
		\tableofcontents[currentsection]
	\end{frame}
	
	\begin{frame}[containsverbatim]
		\frametitle{D�finir un transparent}
		
		\begin{definition}[Beamer]
			\alert{Beamer} est un package qui permet de r�aliser des slides en�\LaTeX. Un slide est repr�sent� par l'environnement frame.
		\end{definition}
		
		\begin{block}{Code (document minimal)}
			\begin{verbatimtab}[3]
\begin{frame}
	\frametitle{Titre}
	
	Contenu
\end{frame}
			\end{verbatimtab}
		\end{block}
	\end{frame}
	
	\section{Modifier le style}
	\begin{frame}
		\frametitle{Modifier le style}
		\only<1>{\framesubtitle{Globalement pour tout le document}}
		\only<2>{\framesubtitle{De mani�re locale}}
		
		\only<1>{
			Plusieurs possibilit�s :
		
			\begin{itemize}
				\item Style pr�d�fini
				\item Red�finitions de commandes
				\item D�finition d'un nouveau style
			\end{itemize}
		}
		\only<2>{
			Il suffit d'utiliser les commandes habituelles de \LaTeX.
		}
	\end{frame}
	
	\section{Conclusion}
	\begin{frame}[allowframebreaks]
		\frametitle{Conclusion}
		
		Bla bla
		
		\framebreak
		
		Suite du blabla
	\end{frame}
\end{document}